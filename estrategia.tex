\section{Estrategia para la transferencia de conocimiento y tecnología}
\begin{evaluacion}
  TOTAL 20\%

  Efectivad y eficiencia de la transferencia de tecnología o conocimientos 

  Estrategía para la implementación o aplicación en Colombia de la
  tecnología o conocimientos internacionales transferidos.
\end{evaluacion}

\begin{instrucciones}
Con un máximo de 200 palabras describa la estrategia que se piensa implementar para la transferencia internacional de conocimientos y tecnología.

  Términos de referencia:
  Pag. 1: fortalecer los servicios de apoyo a la investigación científica.

  Promover la formación de recurso humano para desarrollar las labores de ciencia. 
Cultivar el  capital humano y social de los colombianos residentes en el exterior.

Fortalecer la transferencia de capacidades hacia el país de aquellos colombianos con alta capacidad humana. 

Beneficiarse de la experiencia, conexiones establecidas con instituciones de investigación y conocimiento adquirido por los investigadores colombianos residentes en el exterior.

Fortalecimiento de las redes de trabajo colaborativo
\end{instrucciones}
%1. conociemiento
Involucrando activamente a tres de los más importantes investigadores
Colombianos en física de altas energías que se encuentran
desarrollando sus carreras científicas en el exterior, con Grupos
Colombianos consolidados y emergentes, buscamos desarrollar líneas de
investigación comunes, abrir nuevas líneas y apoyar la formación de
estudiantes. Los participantes de las instituciones extranjeras han
adquirido mucha experiencia en el manejo y optimización de programas
computacionales de física de altas energías que facilitan enormemente
el análisis de modelos de física más allá del modelo estándar. En
particular el Dr. Yaguna es un experto en el manejo del programa
MicroMEGAs, el Dr Bernal tiene mucha experiencia adaptando el programa
SusPect (desarrollado por el Grupo de su director de Doctorado, el
profesor Abdelhak Djouadi) para modelos supersimétricos específicos, y
el Dr. Diego Aristizabal es un experto en programas simbólicos para el
cálculo de procesos a un bucle. La estrategia para aprovechar todo
este potencial consiste en planear visitas y pasantías a lo largo de
todo el proyecto que permitan una interacción directa para facilitar
el intercambio de conocimientos y tecnologías en el área. Se realizará
un evento en la parte final para difundir y potenciar el uso de las
herramientas utilizadas.

%2. Tecnología

%%% Local Variables: 
%%% mode: latex
%%% TeX-master: "proyecto"
%%% End: 
