
\section{Trabajo colaborativo}
\begin{evaluacion}
  TOTAL: 20\% 
  * Trayectoria de colaboración previa entre el colombiano
  en el exterior y el grupo de investigación en Colombia.
  * Vinculación de diversos actores (grupos de investigación) para
  trabajar conjuntamente con la disapora científica en Colombia
  * Estableciemiento de alianzas estrátegicas que existan formalmente
  la institución en Colombia e instituciones de otros paises
  diferentes al de Colombia.
\end{evaluacion}

\subsection{Antecedentes}
\begin{instrucciones}
  En caso de que exista, describa con un máximo de 200 palabras la
  trayectoria de colaboración y alianzas estratégicas formales entre
  el colombiano o institución a la que éste está vinculado en el
  exterior y la entidad ejecutora del proyecto.  
\end{instrucciones}
Los doctores Diego Aristizabal y Carlos Yaguna han venido colaborando
con el Grupo de Fenomenología de Interacciones Fundamentales en los
últimos años en los temas del proyecto. En con el Dr. Aristizabal y el
Dr. Yaguna hemos realizado un trabajado sobre decaimientos exóticos
del Higgs en modelos con rotura de
paridad R~\cite{AristizabalSierra:2008ye}. Con el primero también
hemos estudiado extensiones del Modelo Estándar para explicar física
de neutrinos~\cite{AristizabalSierra:2006ri,Sierra:2008wj} y, en particular, explorado la región de materia oscura tibia en el seesaw
radiativo~\cite{Sierra:2008wj}, y más recientemente hemos propuesto un
modelo de rotura de paridad R para explicar los datos de PAMELA sobre
positrones y antiprotones~\cite{Sierra:2009zq}. El también ha apoyado
la línea de leptogénesis del Grupo colaborando en la formación
doctoral del ahora Profesor Luis Alfredo
Muñoz~\cite{AristizabalSierra:2009bh,AristizabalSierra:2009mq,Sierra:2009bm,AristizabalSierra:2009bh,AristizabalSierra:2007ur,Aristizabal:2003zn}. La
colaboración del Dr. Yaguna nos ha permitido abrir una fructífera
línea de investigación sobre decaimientos de gravitino en modelos con
violación de paridad R~\cite{Choi:2010jt}. El Dr. Nicolás Bernal ha
visitado recientemente nuestro Grupo he impartido el seminario
``Constraining Dark Matter Properties with Gamma-Rays from the
Galactic Center with Fermi-LAT'' y desde entonces se encuentra
colaborando con nosotros en las líneas de investigación propuestas en
éste proyecto.
\subsection{Red de trabajo}
\begin{instrucciones}
  Describa los grupos de investigación o instituciones vinculadas al
  proyecto (dentro o fuera del país), diferentes al colombiano enlace
  en el exterior y a la entidad ejecutora en Colombia. Incluya el rol
  que desempeña cada actor dentro del proyecto.
\end{instrucciones}

\begin{tabular}{|p{6cm}|l|}\hline
Actor & Responsabilidad\\\hline  
EIA Física Teórica y Aplicada 
(Escuela de Ingeniería de Antioquia)& Coinvestigador\\\hline
Grupo de Investigación en Física Teórica, Aplicada y Didáctica
(Instituto Tecnológico Metropolitano, Medellín)&Coinvestigador \\\hline
\end{tabular}

\subsection{Alianzas estratégicas}
\begin{instrucciones}
  En caso de que el proyecto implique la realización de una alianza
  estratégica con una entidad en el exterior, por favor describa
  brevemente en qué consiste y si ya hay algún avance.
\end{instrucciones}
Durante las visitas a realizar a las Universidades de Liege en
Bélgica y Bonn y Muenster en Alemania se realizarán las gestiones para
establecer convenios de cooperación interinstitucionales que faciliten
la movilidad de estudiantes y profesores con las Universidades
participantes en el proyecto de Medellín.
%%% Local Variables: 
%%% mode: latex
%%% TeX-master: "proyecto"
%%% End: 
