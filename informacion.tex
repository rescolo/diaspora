
\section{Información sobre el proyecto}
\begin{evaluacion}
  TOTAL: 30\%

  Coherencia interna y tratamiento particular  de antecedentes,
  justificación, objetivos, metodología y actividades y demás
  información registrada en el formulario.
\end{evaluacion}

\subsection{Sector(es)  en el que se desarrolla el proyecto:}

\subsection{Título:                                        }
\begin{instrucciones}
  El sector puede ser alguno de los propuestos en la convocatoria
  (agricultura, energía, agua, biodiversidad, desarrollo tecnológico e
  innovación) u otros.
\end{instrucciones}
Ciencias Básicas
\subsection{Resumen ejecutivo:                            }
%Máximo 200 palabras.
\subsection{Monto económico total (incluida contrapartida):}
\subsection{Antecedentes:                                  }
Los avances recientes en física de partículas y cosmología han dado
lugar a un entendimiento claro de las tres fronteras a lo largo de la
cual la física de partículas debe avanzar para resolver algunos de los
misterios cruciales de nuestro Universo tales como: el origen de la
masa, la naturaleza de la materia oscura y la energía oscura, la
generación de la asimetría materia-antimateria, y la posible
unificación de las fuerzas. Las tres fronteras que se ilustran en la
Figura~\ref{fig:1}, se han identificado como la Frontera de Energía,
la Frontera de Intensidad, y la Frontera Cósmica \cite{fermilab}. Este
proyecto cubrirá los tres frentes y conectará física de partículas y
cosmología. En el proyecto se estudiarán varios modelos teóricos
confrontándolos con los resultados experimentales recientes y haciendo
predicciones para los experimentos en marcha y los que entrarán
próximamente en funcionamiento.

\begin{figure}
  \centering
\includegraphics[scale=0.3]{three-frontiers-large}
  \caption{Fronteras de energía. Tomado de \cite{fermilab}}
  \label{fig:1}
\end{figure}

Nos encontramos ahora en una época de efervescencia experimental con
detectores de partículas instalados desde las alturas de satélites
artificiales, hasta las profundidades de laboratorios subterráneos a
kilómetros de profundidad. Esto marca una época excitante para la
física de partículas con nuevos datos experimentales disponibles en
las tres fronteras antes mencionadas. La posible correlación de datos
experimentales entre varias de las fronteras podrían permitir un
entendimiento más profundo de los constituyentes del Universo.

La \emph{Frontera de Energía} ha alcanzado la escala del Tera, la
energía a la cual se rompe la simetría electrodébil, con la puesta en
funcionamiento del Large Hadron Collider (LHC). Ubicado en un túnel a
100 metros de profundidad y con una circunferencia de 27 km, el LHC
posee cuatro detectores, dos de los cuales (ATLAS y CMS) están
especialmente diseñados para encontrar el Higgs y señales de nueva física. El LHC
ha comenzado a operar en el 2010, aunque hasta el 2012 lo hará a la
mitad de la energía para la cual fue diseñado. A partir del 2014
aproximadamente, comenzará a funcionar a la energía de diseño de
14~TeV. Para finales del 2012 el LHC habrá completado su primera fase
de operación a una energía de centro de masa de 7~TeV y habrá
acumulado al menos 8~fb$^{-1}$ de datos. Con esta luminosidad se podrá
vislumbrar una escala de energía que hasta ahora no había sido
explorada. La prioridad es la búsqueda de el bosón de Higgs del Modelo Estándar,
el cual a la fecha aún no ha sido encontrado. En el año a venir
se espera obtener las primeras evidencias de su existencia o
por el contrario fortalecer los límites de exclusión hasta un rango de masa
del orden de los 600~GeV.

En el campo de la física de partículas elementales más allá del
Modelo Estándar (ME), quizás el resultado experimental más importante
en los últimos años es el descubrimiento de que los neutrinos
son masivos. Dichas masas han resultado ser pequeñas aunque diferentes de
cero. Las diferencias de masa al cuadrado de los neutrinos, además de sus
correspondientes ángulos de mezcla, son necesarios para poder explicar
las observaciones de oscilaciones de neutrinos a medida que se
propagan sobre grandes distancias. Debido a que los neutrinos
interactúan solo débilmente, los experimentos de neutrinos requieren
de detectores muy masivos y flujos muy
intensos. Los experimentos de neutrinos exploran de esta manera la
\emph{Frontera de Intensidad}. Los experimentos en esta frontera se
enfocan ahora en estudios más precisos de oscilaciones de neutrinos
así como en búsqueda de nuevas fuentes de violación de la simetría CP, mezclas de
sabores de leptones cargados, decaimientos raros, y en la
determinación de la velocidad de propagación de neutrinos altamente
energéticos. Los experimentos que utilizan flujos muy intensos a energías
inferiores que las del LHC pueden proveer información complementaria a los
posibles descubrimientos de los detectores ATLAS y CMS. Un decaimiento
raro que proviene del intercambio de una partícula de gran masa puede
contener información sobre las propiedades del estado intercambiado
aunque este sea demasiado pesado para ser producido directamente.

Recientemente se ha venido acumulando evidencia experimental respecto
al ángulo de mezcla en el sector de neutrinos que aún falta por
determinar, mostrando que dicho ángulo no sólo es diferente de cero
sino que puede ser suficientemente grande como para permitir violación
de CP en el sector leptónico \cite{Schwetz:2011zk}. La violación de CP en el
sector leptónico es un ingrediente necesario para la generación de la asimetría
de materia-antimateria a través del mecanismo de
leptogénesis~\cite{Davidson:2008bu}.


La \emph{Frontera Cósmica} utiliza laboratorios subterráneos,
telescopios basados en tierra y telescopios instalados en satélites
para explorar la componentes oscuras de la materia y la energía, las huellas
de la inflación y el origen y destino del Universo. Las observaciones
de la Frontera Cósmica han alcanzado una precisión mucho mayor de la
podría haber sido imaginada dos décadas atrás. Estos han conseguido
determinar detalles del Universo primitivo los cuales son cada vez más
consistentes con el ``Modelo Estándar'' de Cosmología basado en la
constante cosmológica y en la materia oscura fría ($\Lambda$CDM), que
dan cuenta del 95\% del contenido energético del Universo. Técnicas
novedosas como lentes gravitacionales han aportado significativamente
a nuestro conocimiento del pasado cosmológico, en particular al
aumentar la evidencia experimental de que la materia oscura del
Universo está compuesta de partículas masivas débilmente
interactuantes (WIMPs) \cite{Bertone:2004pz,Jungman:1995df}, formando
halos de materia oscura alrededor de las galaxias.
La materia oscura representa un gran desafío teórico y experimental,
tanto para la astrofísica moderna como para la física de partículas
\cite{Bertone:2004pz, Amsler:2008zzb, Bertone:2010, Jungman:1995df}.
Los WIMPs deben corresponder a nuevas partículas no presentes en el
ME, constituyéndose en la segunda evidencia experimental de necesidad
de física más allá del ME.  Muchos esfuerzos teóricos se han realizado
para construir teorías más allá del ME con candidatos
prometedores de materia oscura. Entre los candidatos a ser WIMPs,
tenemos la más ligera de las partículas supersimétricas, escalares
neutros adicionales, neutrinos derechos, y las partículas de
Kaluza-Klein, cada uno con diferentes implicaciones experimentales.


Las partículas que constituyen la materia oscura típicamente son
consideradas como estables, aunque también podrían ser
inestables siempre y cuando el tiempo de vida media sea mayor que la
edad del Universo. Los WIMPs, si bien es cierto que interactúan débilmente
entre si, eventualmente pueden aniquilarse (o desintegrarse en el caso
de ser inestables) generando productos de aniquilación (o de decaimiento)
que contribuyen a los rayos cósmicos y que pueden llegar a los detectores
instalados en satélites artificiales orbitando la tierra. El valor de múltiples
estudios con detectores de diversos tipos de partículas y radiación
electromagnética sobre un rango muy amplio (incluyendo rayos gamma) se
hace evidente en el conocimiento detallado que se ha logrado alcanzar
y el que se espera mejorar con los experimentos que recientemente han
entrado en funcionamiento.

Cabe mencionar que el programa de detección de materia oscura se ha
centrado en la búsqueda de sus señales en los aceleradores como el LHC
\cite{Baltz:2006fm,Cho:2008tj,Nath:2010zj}, en la detección de la
energía de retroceso en la dispersión de las partículas de materia
oscura con núcleos atómicos
\cite{Green:2007rb,Bertone:2007xj,Drees:2008bv,Green:2008rd}, y la
detección indirecta a través de los estados finales de la aniquilación
y/o decaimiento de la materia oscura
\cite{Bertone:2007aw,Eichler:1989br,Arvanitaki:2008hq,Ibarra:2008jk,Ibarra:2008qg,Buckley:2009kw,Ibarra:2009tn,Ruderman:2009ta},
tales como fotones, neutrinos y la antimateria.


%%%%%%%%%%
El resultado más prometedor en esta área proviene del 2008, cuando
varios experimentos sobre rayos cósmicos como ATIC \cite{:2008zzr} y
los satélites PAMELA \cite{Adriani:2008zr} y Fermi
\cite{Abdo:2009zk}, comenzaron a reportar un exceso en el flujo de
electrones y positrones en rayos cósmicos.
%
Los resultados experimentales muestran un exceso
inesperado en comparación con el background convencional, tanto en el
flujo de electrones más positrones como en la fracción de positrones,
señalando la existencia de una fuente adicional de electrones y
positrones en el halo de la Vía Láctea, mientras que los datos de
antiprotones están de acuerdo con el background astrofísico esperado. 
%
Estos resultados han dado lugar a un sinnúmero de publicaciones
tratando de explicar su origen. Las medidas cada vez más precisas de
rayos gamma por parte de Fermi, pueden ayudar a discernir si el
origen de las anomalías detectadas en electrones y positrones es
debida a fuentes astrofísicas como pulsares cercanos, o a la
aniquilación o el decaimiento de materia oscura.
%
El detector de rayos cósmicos AMS-02~\cite{ams:2009}, ha sido
instalado recientemente en la estación espacial internacional para
medir el espectro y la naturaleza de los rayos cósmicos en un rango de energía
mucho más amplio y con una estadística mucho mejor que PAMELA.

Los experimentos de detección directa de materia oscura instalados en
laboratorios subterráneos como XENON100 \cite{Aprile:2011ts}
o CDMS \cite{Ahmed:2009zw,Ahmed:2010wy}, han
comenzado a explorar las regiones predichas por algunos de los modelos
más estudiados de materia oscura.
Sin embargo, hasta ahora dichos detectores no han encontrado señales
de dichas partículas. Por el contrario, existen otras medidas tomadas
con detectores tales como DAMA/LIBRA~\cite{Bernabei:2010mq},
CoGent~\cite{Aalseth:2011wp} o CRESST-II~\cite{Angloher:2011uu} que
favorecerían la existencia de partículas de materia oscura con una masa
en el rango de 6 a 15 GeV.
La situación experimental está lejos de ser sencilla, ya que por el momento
no es claro cómo conciliar los diferentes resultados de la detección directa.
No obstante, una nueva generación de detectores con masas del orden de la
tonelada como XENON1T o SuperCDMS, o con técnicas más avanzadas de
discriminación del ruido de fondo como MIMAC \cite{Billard:2011yf}
o DRIFT~\cite{Pipe:2010zz} en el caso de la detección direccional estará
próximamente disponible para así poder resolver este conflicto.
En definitiva, para los próximos años se espera una gran actividad en
el campo de la detección directa de materia oscura que permitirá
testear una porción considerable del espacio de parámetros favorecido por
la materia oscura del Modelo Estándar supersimétrico mínimo restringido
(cMSSM por sus siglas en inglés).\\ 
%%%%%%

Supersimetría \cite{Martin:1997ns,Haber:1984rc} es una de las teorías
propuestas para resolver el problema de la jerarquía del sector de Higgs
en el ME. El Modelo Estándar supersimétrico mínimo (MSSM) es, como su
nombre lo indica, la extensión supersimétrica más simple del ME.
Dicho modelo es mínimo en el sentido que contiene una mínima cantidad de campos
e interacciones.
Todas las interacciones del modelo están especificadas por las simetrías
(gauge, Lorentz, supersimetría\dots) y el superpotencial,
el cual debe ser una función holomórfica de los campos escalares del
modelo. El MSSM, además de tener la virtud de resolver el problema de jerarquía
estabilizando la masa del Higgs, unifica los acoplamientos gauge a una escala
llamada de Gran Unificación y proporciona un candidato viable para la materia
oscura. Este corresponde a la partícula supersimétrica más ligera, el
cual es estable cuando se supone la conservación de una simetría $Z_2$,
conocida también en este caso como paridad R~\cite{Ellis:1983ew} ($R_p$).
En el ME supersimétrico, además de los términos
del superpotencial que constituyen la semilla del potencial escalar y
del lagrangiano de Yukawa, las simetrías del modelo permiten los siguientes
términos:
\begin{equation}
  \label{eq:5}
  W_{\cancel{R_p}} = \mu_i\widehat{L}_i\widehat{H}_u + 
  \lambda_{i j k}\widehat{L}_i\widehat{L}_j\widehat{l}_k +
  \lambda'_{i j k}\widehat{L}_i\widehat{Q}_j\widehat{d}_k + 
  \lambda''_{ijk}\widehat{u}_i\widehat{d}_j\widehat{d}_k\,.
\end{equation}
% La simetría paridad R prohíbe todos estos términos bilineales y trilineales. El modelo resultante es conocido como modelo estándar supersimétrico con mínimo número de operadores (MSSM). 
% Además de tener la virtud de unificar a los acoplamientos gauge del
% Modelo Estándar, la supersimetría a bajas energías proporciona un
% candidato a materia oscura que corresponde a la partícula
% supersimétrica más ligera, la cual es estable cuando se supone la
% conservación de la simetría de paridad R \cite{Ellis:1983ew}.
La conservación de la simetría de paridad R prohíbe la aparición
\emph{simultánea} de operadores que violan el número bariónico y
leptónico, $\lambda'$ y $\lambda''$, evitando así el decaimiento
prematuro del protón. El modelo resultante es conocido como Modelo Estándar supersimétrico con mínimo
número de operadores (MSSM). Otras simetrías similares a paridad R
que permiten bien sea la presencia sólo de operadores que violan número
leptónico o número bariónico también dan lugar a un protón
suficientemente estable. Estos modelos fenomenológicamente viables son
conocidos en general como modelos con violación de paridad R. Cuando
paridad R no se conserva en el Modelo Estándar supersimétrico
\cite{Barbier:2004ez}, estos operadores pueden conducir al decaimiento
de la partícula supersimétrica más ligera, teniendo profundas
implicaciones en la búsqueda de supersimetría en aceleradores y en los
experimentos de detección directa e indirecta de materia oscura. El
Modelo Estándar supersimétrico con rotura bilineal de paridad R
(BRpV)~\cite{Diaz:1997xc,Hirsch:2000ef,Diaz:2003as}, es la versión más simple que da cuenta de las masas y
mezclas de neutrinos sin necesidad de introducir partículas
adicionales. 

Con la entrada en funcionamiento del LHC se han estado excluyendo
partes importantes de modelos supersimétricos basado en señales de
energía faltante. En general, en modelos con rotura de paridad R hay
una degradación de esta señal haciendo más difícil su búsqueda en
colisionadores hadrónicos. En modelos con rotura de paridad R a través
de operadores que violan número bariónico, supersimetría puede quedar
incluso oculta por el background de QCD. De modo que se espera que los
modelos con rotura de paridad R ganen cada vez más importancia a
medida que se continué excluyendo el MSSM~\cite{Bomark:2011fj}.

Cuando la supersimetría se promueve a ser una simetría local de la
naturaleza, la teoría de supergravedad resultante requiere un
supermultiplete que incluya el gravitón y su supercompañero, el
gravitino~\cite{Martin:1997ns,Nilles:1983ge}. El gravitino ($\tilde
G$) adquiere masa a partir de la ruptura espontánea de la
supersimetría (llamado súper mecanismo de Higgs), la cual viene dada
por $m_{\tilde G}=\langle F\rangle/M_P$, donde $\langle F\rangle$ es el valor esperado de vacío
del campo auxiliar que rompe la supersimetría, y $M_P$ es la masa de
Planck. Por lo tanto, dependiendo del escenario de rotura de
supersimetría, el rango de la masa del gravitino va desde los eV hasta
mas allá de la escala del TeV.

La aparición natural del gravitino en supergravedad trae un problema
con el escenario de la cosmología estándar. En el Universo primitivo,
cuando el gravitino está en equilibrio térmico con el plasma, la
densidad reliquia de gravitinos es mayor que la densidad crítica, lo
que implica que el Universo podría auto-contraerse. Por lo tanto, es
fundamental invocar una fase inflacionaria en el Universo con el
objetivo de diluir la densidad reliquia de gravitinos. Sin embargo,
los gravitinos también se producen en la fase de recalentamiento (a
través del decaimiento del inflatón) después de la inflación. Durante
o después de la nucleosíntesis, el decaimiento de la segunda partícula
supersimétrica más ligera o  podría
generar una lluvia electromagnética y de hadrones que echarían a
perder la predicción exitosa de las abundancias de elementos ligeros
\cite{Sarkar:1995dd}. La solución a este problema se obtiene mediante
la introducción de una violación de paridad R
\cite{Takayama:2000uz, Buchmuller:2007ui}, la cual permite a la segunda partícula
supersimétrica más ligera decaer en las partículas del ME
antes de la nucleosíntesis, y permite a los restantes gravitinos tener
un tiempo de vida mayor que la edad del Universo. En este último caso,
el gran tiempo de vida que se requiere para un candidato de materia
oscura inestable se obtiene gracias a las débiles interacciones del
gravitino, las cuales son suprimidas por la escala de Planck y por los
pequeños acoplamientos bilineales ó trilineales que violan paridad
R. Si las masas de neutrinos son explicadas a través del mecanismo de
seesaw, el escenario con el gravitino como materia oscura y con
violación de paridad R también se ve favorecido por la teoría que
explica la asimetría entre materia y antimateria, la leptogénesis, ya
que alivia la tensión entre la alta temperatura de recalentamiento
requerida por leptogénesis y las restricciones provenientes de la
teoría de la Nucleosíntesis.  Por lo tanto, en modelos con violación de
paridad R, el gravitino como materia oscura producido en el Universo
temprano es viable y bien motivado.

Con respecto a la detección, cuando la simetría paridad R se conserva,
los gravitinos tienen interacciones muy débiles y por lo tanto ninguna
señal de la materia oscura gravitino se puede observar en los
experimentos de detección directa o indirecta. En cuanto a las señales
en los colisionadores, la producción directa de gravitinos esta muy
suprimida, pero la segunda partícula supersimétrica más ligera puede
dejar rastros en los detectores. Por otro lado, si el gravitino es
inestable (debido a la violación de paridad R) sus productos de
desintegración puede llevar a señales observables en las búsquedas
indirectas de materia oscura
\cite{Bertone:2007aw,Ibarra:2007wg,Covi:2008jy,Ibarra:2008qg}. Los
gravitinos puede ser detectados indirectamente a través del
decaimiento a rayos gamma, neutrinos o antimateria a través de los
experimentos descritos en la Frontera Cósmica. De particular interés
se tiene la región en la cual la masa del gravitino es menor que $80$
GeV, ya que el gravitino puede decaer únicamente a un neutrino y un
rayo gamma, produciendo lo que se conoce como una línea de rayos gamma
(un fotón monoenergético). Los datos del espectro de rayos gamma
extragaláctico, tomados por los satélite EGRET y Fermi, ponen fuertes
cotas sobre el espacio de parámetros de estos modelos. De hecho, en la
referencia \cite{Yuksel:2007dr} se obtuvieron restricciones sobre el
tiempo de vida del candidato a materia oscura en los modelos
supersimétricos con violación de paridad R, para un rango de masas de
$10^{-5}$ a $10$ GeV. En esta misma dirección, en la referencia
\cite{Vertongen:2011mu} se realizó una búsqueda sistemática de señales
de líneas de rayos gamma en dichos modelos, sin encontrar evidencia de
estas. Con este resultado, se obtuvieron restricciones sobre el tiempo
de vida del candidato a materia oscura para una masas entre
$2<m_{DM}< 600$ GeV. Como hemos mostramos en \cite{Choi:2010jt},
y ratificado por otro grupo con datos más recientes en
\cite{Garny:2010eg}, los últimos datos de líneas de rayos gamma
publicado por Fermi excluyen la posibilidad de que las masas de
neutrinos puedan ser generadas por términos bilineales de ruptura de
paridad R si la temperatura de recalentamiento está sobre $10^9\ $GeV
como sugiere leptogénesis.  En esta región del espacio de parámetros
el gravitino es mayor de unos 10 GeV y los acoplamientos bilineales
son tan pequeños que el decaimiento de la partícula siguiente a la
LSP, la NLSP (de sus siglas en inglés), ocurre fuera del detector en
el LHC, por lo que el modelo es básicamente indistinguible del
MSSM. En este caso los datos de detección indirecta de materia oscura
a través de rayos cósmicos serían la única forma de diferenciar el
modelo del MSSM (donde el neutralino es el candidato de materia
oscura). Sin embargo, si consideremos masas de gravitinos más
pequeñas, aunque ya no podríamos explicar bariogénesis a través de
leptogénesis, recuperamos la posibilidad de explicar las masas y
mezclas de los neutrinos a través de los operadores bilineales y
trilineales que rompen paridad R~\cite{Hirsch:2005ag}, un mecanismo,
que a diferencia del seesaw, si se puede verificar en el LHC.
En modelos supersimétricos con violación bilineal de paridad R también
la explicación de las masas y mezclas de neutrinos es especialmente
simple \cite{Hirsch:2000ef,Diaz:2003as,Hirsch:2004he,Hirsch:2008ur}: Por un lado se encuentra la
generación a nivel árbol de la escala de masa atmosférica, el ángulo
de mezcla atmosférico y el ángulo de reactor. En el otro lado, la
escala de masa solar y el ángulo de mezcla solar se obtienen por medio
de las correcciones cuánticas a unbucle de la matriz de masa de los
neutrinos a nivel árbol.

La metodología desarrollada en el estudio exhaustivo de modelos
supersimétricos la hemos logrado aplicar a otros modelos de generación
radiativa de masas de neutrinos con nuevas partículas a la escala del
TeV asequibles en el LHC
\cite{Sierra:2008wj,AristizabalSierra:2006ri}.
El método para obtener una partícula estable del modelo supersimétrico
a partir de una simetría $Z_2$ como el caso de paridad R, ha sido
usado recientemente para obtener extensiones del Modelo Estándar que
explican materia oscura. De hecho, en el programa computacional
MicrOMEGAs~\cite{Belanger:2006is} se puede implementar cualquier
extensión del Modelo Estándar que posea una simetría $Z_2$ para
calcular numéricamente la densidad de reliquia de materia oscura y la
sección eficaz WIMP--nucleón, relevante para los experimentos de
detección directa. En extensiones del Modelo Estándar que implementan
una simetría $Z_2$ con un singlete escalar
neutro~\cite{McDonald:1993ex,Burgess:2000yq,Davoudiasl:2004be,Barger:2007im,Dick:2008ah,Yaguna:2008hd,Goudelis:2009zz,Yaguna:2011qn} o un doblete
inerte escalar~\cite{Barbieri:2006dq,Majumdar:2006nt,Hambye:2007vf,LopezHonorez:2006gr,Gustafsson:2007pc,Agrawal:2008xz,Andreas:2009hj,Nezri:2009jd,Lundstrom:2008ai,Dolle:2009ft}, la partícula impar neutra y más liviana
constituye un candidato viable de materia oscura estable escalar con
implicaciones en aceleradores y en experimentos de detección
directa e indirecta de materia
oscura. En el caso de la extensión del Modelo Estándar con un doblete
inerte se puede tener una conexión con la frontera de intensidad si se
adicionan neutrinos derechos impares bajo la simetría $Z_2$. A
diferencia del mecanismo seesaw tradicional, en este caso
correspondiente al seesaw radiativo~\cite{Ma:2006km,Kubo:2006yx,Kubo:2006rm,Suematsu:2009ww,Gelmini:2009xd}, las masas de los
neutrinos ligeros se generan al nivel de un bucle de modo que los
neutrinos derechos pueden estar en la escala del TeV con yukawas
suficientemente grandes como para ser producidos en el
LHC~\cite{Sierra:2008wj}. En general, en modelos en los cuales es
posible explicar la masa de los neutrinos total o parcialmente por
métodos radiativos, no sólo es posible dar cuenta de la pequeñez de
sus masas con respecto a la de los otros fermiones, sino también,
hacer predicciones muy concretas en aceleradores de partículas, como
el LHC.


Las observaciones astronómicas sugieren que el Universo está compuesto
en su mayor parte de materia. En el contexto del Big Bang, esto
implica que en algún momento grandes cantidades de materia y
antimateria se aniquilaron dejando el pequeño exceso de materia que
constituye el Universo observable actual. El problema de explicar el
exceso inicial de materia sobre antimateria se conoce con el nombre de
bariogénesis. Dentro del Modelo Estándar, aunque contiene los
ingredientes necesarios, no es posible explicar bariogénesis. Los
modelos con neutrinos derechos contienen todos los ingredientes para
poder explicar bariogénesis a través de leptogénesis~\cite{}.




\subsection{Justificación:                                 }
\begin{instrucciones}
  CODI: 

  * ¿Está bien definido el problema que se quiere investigar?:
  Fenomenología de modelos más allá del ME motivados por evidencias
  fenomenológicas del ME.  
  * ¿Es clara su justificación desde el punto de vista académico,
  científico, tecnológico, social, económico y legal? (15).
  científico: contribuir a la correspondiente rama del conocimiento

  COLCI: en este ítem usted deberá describir de forma precisa y completa la
  
  * naturaleza y magnitud del problema de investigación que se quiere
  abordar:
  ** Construir modelos nuevos.
  ** Explorar modelos existentes.
  Formule claramente las preguntas concretas a las cuales se
  quiere responder en el contexto del problema planteado.
\end{instrucciones}
%Tesis
El Grupo de Fenomenología de Interacciones Fundamentales de la
Universidad de Antioquia, en conjunto con Grupos emergentes en
otras instituciones de la región conformados por egresados de
doctorado de nuestro Grupo, se ha enfocado en la investigación
científica de aspectos fenomenológicos en las tres fronteras de la
física de altas energías y la cosmología. Estas investigaciones se han
hecho en colaboración con investigadores internacionales, en
particular con científicos colombianos trabajando en el exterior que
participan en este
proyecto~\cite{Sierra:2009zq,Sierra:2008wj,AristizabalSierra:2008ye,Choi:2010jt,AristizabalSierra:2009bh};
mediante el cual buscamos dar continuidad a estos desarrollos a través
de investigaciones de impacto en la comunidad científica de estas
áreas de frontera. En este proyecto se continuarán explorando
diferentes extensiones del Modelo Estándar que explican las masas y
mezclas de neutrinos, y que contienen un candidato viable de materia oscura.
Además se harán predicciones concretas para los experimentos presentes y futuros
en las tres fronteras de la física de partículas y la cosmología para los
diferentes modelos en cuestión.

Con este proyecto queremos beneficiarnos de la experiencia y capacidad
de reconocidos científicos colombianos que trabajan en instituciones
de primera línea en el exterior para ampliar los horizontes
investigativos del país y poder participar más activamente en las
fronteras de investigación de la física de partículas y cosmología.


Hemos estudiado exhaustivamente las predicciones del modelo con
ruptura bilineal de paridad R, BRpV, que incluye sólo los tres
operadores proporcionales a $\mu_i$, tanto para el Tevatron como
para el LHC, asumiendo que el neutralino es la partícula
supersimétrica más liviana (LSP)
\cite{Magro:2003zb,deCampos:2005ri,deCampos:2007bn,deCampos:2008ic,deCampos:2008re,DeCampos:2010yu}. Dichas
búsquedas ya han comenzado a implementarse por l colaboración ATLAS
del LHC~\cite{:2011iu} donde se muestra el espació de parámetros
excluido en el cMSSM con BRpV para una luminosidad de $1\
\text{fb}^{-1}$.  En la referencia~\cite{DeCampos:2010yu} nos hemos enfocado en la
intersección entre la Frontera de Energía y la Frontera de Intensidad
al determinar el nivel de precisión con el que se puede llegar a medir
en el LHC la correlación entre decaimientos de neutralinos a muón y
tau con el ángulo de mezcla atmosférico de neutrinos: una predicción
muy concreta que de no observarse en el LHC en los próximos años
descartaría completamente un modelo como el del mecanismo de generación de
masas para neutrinos. 

Cuando el gravitino es la LSP, se constituye en un candidato viable de
materia oscura inestable. Este caso ha sido estudiado exhaustivamente
en la literatura por sus implicaciones en experimentos de rayos
cósmicos, especialmente en el caso en que la temperatura de recalentamiento es mayor que $T_R\gtrsim 10^9$~GeV 
\cite{Takayama:2000uz,Buchmuller:2007ui,Bertone:2007aw,Ibarra:2007wg,Ishiwata:2008cu,Covi:2008jy,Ibarra:2008qg}. 
En tal caso el modelo también explica bariogénesis a través de leptogénesis. 


También hemos explorado~\cite{Restrepo:2011rj} la intersección en
las tres fronteras encontrando la restricciones en el modelo BRpV
provenientes de las medidas de líneas de líneas de rayos gamma
obtenidas por Fermi cuando el gravitino es la partícula
supersimétrica más liviana y da cuenta de la materia oscura del
Universo. Como los acoplamientos que violan paridad R deben ser
suficientemente grandes para explicar masas y mezclas de neutrinos,
los decaimientos que violan paridad R de la partícula supersimétrica
siguiente a la más liviana (NLSP) dominan sobre los canales de
decaimiento a gravitino, de modo que la fenomenología en aceleradores,
incluyendo las correlaciones entre las fronteras de energía e
intensidad, estudiada previamente se sigue manteniendo intacta.

Esto hace los modelos con violación de paridad R con el gravitino como
materia oscura especialmente interesantes pues genera predicciones en
la tres fronteras de las física de partículas y la cosmología, las
cuales se han venido confrontando con resultados experimentales
muy diversos. Es así como por ejemplo se ha logrado restringir hasta
en 12 órdenes de magnitud el tiempo de vida media del gravitino, desde
los $10^{17}$ s (correspondientes a la edad del Universo) hasta los
$10^{29}$ s, en el rango de masas alrededor de 10~GeV, cuando el
gravitino decae dominantemente a neutrino--fotón~\cite{Vertongen:2011mu}.  

\begin{gravitinodm}

  Diversos estudios han considerado las restricciones que las medidas
  de rayos cósmicos imponen sobre modelos supersimétricos con violación de
  paridad R. En el caso de operadores trilineales que violan número
  leptónico se han venido reportando las restricciones sobre la masa del
  gravitino a partir de los datos de rayos cósmicos considerando sus
  decaimientos a tres cuerpos, y a un bucle mediados por partículas
  gauge~\cite{Lola:2007rw,Lola:2008bk,Bomark:2009zm}. En este proyecto extenderemos esos análisis teniendo en
  cuenta que los operadores trilineales de rotura de paridad R pueden
  inducir operadores bilineales de rotura de paridad R. En tal caso el
  gravitino tiene un nuevo canal a nivel árbol a neutrino--fotón que
  afectarán las cotas sobre la masa del gravitino especialmente en la
  región por debajo de los 80 GeV. 

  También estudiaremos las restricciones sobre la masa del gravitino
  cuando los operadores trilineales son usado para generar las masas y
  mezclas de los neutrinos, extendiendo el análisis realizado
  previamente en la referencia~\cite{Bajc:2010qj}.

En el marco de los modelos supersimétricos con violación bilineal de paridad R y materia oscura de gravitinos, se supone que el gravitino es la más ligera de las partículas supersimétricas y que da cuenta de la densidad de materia oscura observada del Universo. En estos modelos, el gravitino es inestable debido a la rotura de paridad R y puede decaer en partículas del Modelo Estándar a través de las interacciones bilineales que rompen paridad R. En este escenario, todos los efectos que violan paridad R, incluyendo el decaimiento del gravitino y las masas diferentes de cero de los neutrinos, son controlados por los acoplamientos bilineales $\xi_i$. 

La detección indirecta de materia oscura involucra la búsqueda de los estados finales en las desintegraciones del gravitino, entre los que se encuentran los rayos gamma. A partir de las restricciones sobre el tiempo de vida del gravitino se pueden obtener restricciones en función de la masa del gravitino para el acoplamiento bilineal dominante $\xi_3$. Cuando se asume la unificación de las masas de los gauginos a la escala de gran unificación, la restricción resultante sobre $\xi_3$ es muy fuerte: $\xi_3\lesssim 10^{-7}$ para $m_{\tilde G}\gtrsim 1$ GeV.  Sin embargo, la unificación de las masas de los gauginos no es una condición que se deba asumir necesariamente en los modelos supersimétricos, lo cual implicaría que la restricción sobre $\xi_3$ puede ser relajada.

Nuestro grupo \cite{Nardi:2008ix} fue uno de los
primeros en proponer una explicación en términos de materia oscura
inestable para explicar el exceso de positrones observado por el
satélite PAMELA en el 2008 \cite{Adriani:2008zr}. Luego hemos
construido un modelo basado en supersimetría con ruptura de paridad R
a través de términos trilineales del tipo $\lambda$, caso
{\it ii}, para explicar la preferencia por decaimientos leptónicos
de la partícula de materia oscura, que en este caso es el
neutralino \cite{Sierra:2009zq}.
\end{gravitinodm}

\begin{bbrpvlhc}
  Existen varias extensiones del modelo BRpV que adicionan nuevas
  partículas tales como las resultantes de la adición de un
  supertriplete escalar de Higgs~\cite{}, o singletes adicionales para
  explicar el problema del término $\mu$. Estas extensiones se podrían
  parametrizar en función de operadores no renormalizables que se
  podrían adicionar para tener un marco general del modelo más allá del
  BRpV. El Dr. Nicolás Bernal es un experto en extensiones más allá
  del MSSM para parametrizar el sector de Higgs~\cite{Bernal:2007uv,Bernal:2009hd}
  y será de gran ayuda en esta área.
\end{bbrpvlhc}

\begin{brpvlhc}
  En este proyecto pretendemos seguir explorando más correlaciones
  de observables en el LHC con física de neutrinos para determinar con
  que nivel de precisión se podrían llegar a medir en el LHC. Cuando
  el neutralino es la LSP, los decaimientos a tres cuerpos mediados
  por sfermiones con muones y electrones en los estados finales, están
  correlaciones con el ángulo de mezcla solar, y la longitud de
  decaimiento del neutralino está correlacionada con la diferencia de
  masa atmosférica.
\end{brpvlhc}

%Frontera de energía: que esta hecho y que falta por hacer
%Hablar del modelo bilineal y su extensión como modelo efectivo

\begin{darkmatter}
  El cálculo de la densidad de neutralinos en supersimetría se conoce
  a un bucle, lo cual es necesario para determinar más precisamente las
  regiones de exclusión obtenidas a partir de las medidas de detección
  directa de materia oscura. En el marco de este proyecto,
  con la participación del estudiante de Maestría, y con el
  asesoramiento del Dr. Carlos Yaguna, se 
  calculará las contribuciones a un bucle para la sección eficaz de la
  partícula escalar de materia oscura en varios  modelos con materia oscura
  escalar impar bajo una simetría $Z_2$.

  En este proyecto se extenderá el estudio realizado
  en~\cite{Sierra:2008wj} con materia oscura tibia en el modelo seesaw
  radiativo en la intersección de las fronteras de energía e
  intensidad, para estudiar las correlaciones entre mezclas y masas de
  neutrinos con combinaciones de branchings de decaimientos de las
  partículas impares de dicho modelo en el caso de materia oscura
  fría. Para las señales más representativas realizaremos las
  simulaciones computacionales para el detector ATLAS del LHC.
\end{darkmatter}



\begin{leptogenesis}
  Aprovechando la experiencia del Dr. Diego Aristizábal en el área,
  también se explorará la posibilidad de generar leptogénesis e
  implementar simetrías no abelianas tipo $A_4$ en el sector de Yukawa
  del seesaw radiativo para obtener ángulos de mezcla $\theta_{13}$
  suficientemente grande como sugieren las medidas experimentales
  actuales.
\end{leptogenesis}

Un modelo ideal sería uno que de cuenta de las masas y mezclas de
neutrinos, tenga un candidato de materia oscura que sirva para explicar
el exceso de positrones en experimentos de rayos cósmicos y a la vez
contenga los ingredientes para generar bariogénesis.  En este
proyecto pretendemos formular modelos de estas características, además
de continuar explorando otros posibles modelos que puedan dar cuenta
al menos de las dos evidencias fenomenológicas más importantes: masas
de neutrinos y materia oscura, las cuales requieren una extensión del
Modelo Estándar.


Con base en lo planteado anteriormente, lo que proponemos en este
proyecto es tratar de responder la siguiente pregunta: ¿Cuáles serían
las restricciones impuestas por los resultados experimentales
presentes y futuros en las tres fronteras de la física de partículas y
cosmologías sobre modelos que presenten una partícula candidata a
materia oscura y la vez generen masas para los neutrinos?
 

\subsection{Objetivos:                                     }
Utilizar los resultados experimentales en las fronteras de energía,
intensidad y cósmica para restringir modelos existentes de masas y
mezclas de neutrinos que posean algún candidato de materia oscura.

\subsubsection{Objetivos específicos}
\label{sec:objet-espec}

\begin{enumerate}
\item Consolidar y ampliar las líneas de investigación en las tres
  fronteras de la física de partículas y cosmología en el país, a
  través de la producción de artículos internacionales, aprovechando la
  experiencia y capacidades de los investigadores colombianos
  trabajando en el exterior que participan en el proyecto.

\item \begin{gravitinodm}
  Construir un programa computacional que incluya todos los
  decaimientos del gravitino en Modelo Estándar supersimétrico con
  rotura de paridad R a través de operadores con violación de número
  leptónico, incluyendo los decaimientos a dos y tres cuerpos de los
  operadores bilineales inducidos por el correspondiente término
  trilineal cuando evoluciona desde la escala de unificación en un
  contexto de supergravedad mínima.
\end{gravitinodm}
\label{item:gravitinodm1}

\item 
\begin{gravitinodm}
  Extender el programa computacional de espectro supersimétricos
  SuSpect~\cite{Djouadi:2002ze} para calcular el espectro supersimétrico y sus
  correspondientes ángulos de mezcla cuando la escala de los
  sfermiones esté varios ordenes de magnitud por encima de la escala
  electrodébil, para luego determinar la dependencia de las masas
  de neutrino inducidas por operadores trilineales con rotura de
  paridad R con la escala de los sfermiones.
\end{gravitinodm}
\label{item:gravitinodm2}
\item 
\begin{gravitinodm}
  A partir de las medidas proporcionadas por experimentos de rayos cósmicos, como los
  de líneas de rayos gamma de Fermi, determinar las restricciones
  en el espacio de parámetros del modelo estándar supersimétrico con
  rotura de paridad R a través de operadores bilineales y trilineales
  con violación de número leptónico, estableciendo las señales
  dominantes en cada rango de masas de gravitinos.
\end{gravitinodm}

\item
\begin{gravitinodm}
  Estudiar cuales serían las implicaciones en física de neutrinos
  cuando no se consideran las restricciones de leptogénesis en con
  rotura de paridad R a través de operadores con violación de número
  leptónico y con el gravitino como materia oscura.
\end{gravitinodm}

\item 
\begin{bbrpvlhc}
  Calcular las masas y mezclas de neutrinos en la extensión del modelo
  BRpV con operadores no renormalizables de dimensión cinco con
  violación de número leptónico, sus consecuencias y restricciones
  experimentales en las tres fronteras de la física de partículas y la
  cosmología.
\end{bbrpvlhc}
\label{item:bbrpvlhc1}
\item 
\begin{brpvlhc}  
  En el modelo con ruptura bilineal de paridad R con el neutralino
  como la LSP, determinar con que nivel de precisión se podría llegar
  a medir en el LHC la correlación entre la diferencia de masa
  atmosférica al cuadrado y la longitud de decaimiento del neutralino.
\end{brpvlhc}
\label{item:bbrpvlhc2}
\item 
\begin{darkmatter}
  Calcular la sección eficaz materia oscura--nucleón a un bucle en el
  seesaw radiativo tanto en el caso materia oscura escalar como
  fermiónica.
\end{darkmatter}

\item 
\begin{darkmatter}
  Establecer las regiones del espacio de parámetros en el seesaw
  radiativo que dan lugar a una densidad reliquia apropiada
  de materia oscura y que satisfacen las restricciones
  experimentales de masas y mezclas de neutrinos. Para cada región
  establecer las señales que se esperan en el LHC y en experimentos de
  detección directa de materia oscura.  Para las señales más
  representativas, hacer simulaciones para el detector ATLAS del LHC.
\end{darkmatter}
\label{item:darkmatter1}

\item 
\begin{darkmatter}
  Implementar simetrías no Abelianas para simplificar el sector de
  Yukawas de los neutrinos derechos para determinar las condiciones
  bajo las cuales se puede tener un ángulo de mezcla $\theta_{13}$
  grande y estudiar sus implicaciones para la generación de
  leptogénesis en el seesaw radiativo.
\end{darkmatter}
\label{item:darkmatter2}

\item Mejorar la infraestructura computacional de alto rendimiento en
  la Universidad de Antioquia, la región y el país, con la conexión en
  grid de los cluster existentes, su vinculación a Grid Colombia, y la
  implementación de la arquitectura CUDA para GPUs con la compra de un
  nodo para blade server con puertos para tarjetas gráficas Tesla de
  múltiples nodos.

\end{enumerate}

\subsection{Metodología:                                   }
\begin{instrucciones}
  Esta debe reflejar la estructura lógica y el rigor científico del
  proceso de investigación, empezando por la elección de un enfoque
  metodológico específico y finalizando con la forma como se van a
  analizar, interpretar y presentar los resultados.
\end{instrucciones}
%%%

%Final
A continuación se presenta la metodología a seguir a la hora de
analizar las predicciones, o establecer las restricciones, de un
modelo de nueva física que explique las masas y mezclas de neutrinos,
o tenga un candidato de materia oscura estable o inestable. Para
ello usaremos como paradigma el trabajo que hemos venido realizando
con modelos que contienen ruptura bilineal de paridad R y ejemplificando cada
paso con referencias concretas.
\begin{enumerate}
\item Se construye un modelo que solucione un problema fenomenológico
  del Modelo Estándar, como el problema de las masas de neutrinos
  \cite{Hirsch:2000ef}, o la materia oscura (o ambos
  \cite{Hirsch:2005ag,Restrepo:2011rj}), mostrando bajo qué mecanismo
  específico se soluciona el problema en cuestión. La solución del
  problema requiere introducir partículas adicionales las cuales
  pueden buscarse en detectores de partículas en las fronteras de
  energía, de intensidad o cósmica.
  \label{item:5}
\item Se determina el espacio de parámetros compatible con los datos
  experimentales, se calculan los branchings de decaimiento y las
  secciones eficaces y se establecen correlaciones entre estos
  observables en la frontera de energía con otros observables en la
  frontera de intensidad y cósmica. En el caso de modelos con
  mecanismos de generación radiativa de masas de neutrinos las
  correlaciones se buscan con los datos de oscilaciones de neutrinos
  \cite{Diaz:2003as}, y en el caso de materia oscura con experimentos
  de detección directa e indirecta \cite{Choi:2010xn,Restrepo:2011rj}.
  \label{item:6}
\item Se desarrolla un programa computacional que dado los parámetros
  de entrada del modelo entregue las secciones eficaces y los
  branching de decaimiento en un formato adecuado (el SLHA
  \cite{Allanach:2008qq}) para ser usado luego por programas de Montecarlo
  de generación de eventos como PYTHIA \cite{Sjostrand:2006za}. El
  modelo con ruptura bilineal de paridad R ha sido implementado en el
  programa computacional SPheno~\cite{Porod:2003um} (el cual tiene
  implementado el formato de salida SLHA versión 2). Modelos con un
  candidato de materia oscura estable bajo una simetría $Z_2$ pueden
  ser implementados en MicrOMEGAs~\cite{Belanger:2006is}.
  \label{item:7}
\item Se divide el espacio de n-parámetros establecido en el paso
  \ref{item:6} en una malla n-dimensional. Para cada punto se corre el
  programa computacional desarrollado en el paso \ref{item:7} y los
  correspondiente datos de salida se pasan a PYTHIA usando la
  interfase SLHA. En cada punto se realiza con PYTHIA una simulación
  que consiste en generar aleatoriamente eventos de acuerdo a la
  geometría y características de detectores específicos, para
  determinar la factibilidad de descubrir las señales en aceleradores
  \cite{Magro:2003zb,deCampos:2005ri,deCampos:2007bn,deCampos:2008ic,deCampos:2008re},
  o en experimentos de detección indirecta \cite{Choi:2010xn} de
  materia oscura. En el caso de detección directa el número esperado
  de eventos puede calcularse directamente con MicrOMEGAs. Este paso
  requiere normalmente herramientas de computación distribuida en
  clusters de computadores.
  \label{item:8}
\item Como modelos muy diferentes pueden dar lugar a las mismas
  señales en detectores, se debe hacer también simulaciones en PYTHIA
  del nivel de precisión con el que se pueden determinar observables
  en los experimentos de la frontera de energía que se puedan
  correlacionar con observables en los experimentos de las fronteras
  de intensidad y cósmica.
  experimentos.
  \label{item:9}
\end{enumerate}

Las simulaciones se realizan con el fin de desarrollar todas las
herramientas necesarias para que los grupos experimentales de los
aceleradores puedan comparar los datos obtenidos con modelos
específicos y puedan descubrir o poner cotas sobre las nuevas
partículas propuestas. En el caso de detección directa o indirecta de
materia oscura donde las señales están preestablecidas, la
simulaciones se realizan para poder comparar con los datos obtenidos o
que se pueden llegar a obtener. Es así como el trabajo fenomenológico
realizado alrededor del modelo BRpV ya ha dado sus frutos al estar ya
en las agendas de búsqueda de los grupos experimentales del
Tevatron~\cite{Brigliadori:2008vf} y el LHC~\cite{:2011iu}. Esperamos
que con este proyecto se pueda llegar al mismo nivel de relación de
teoría y experimento con otras extensiones del Modelo Estándar que
también explican las masas y mezclas de neutrinos y tienen candidatos
de materia oscura.



\begin{gravitinodm}
  En el modelo donde la ruptura de paridad R a través de operadores
  que violan número leptónico puede explicar la densidad de reliquia
  de materia oscura cuando la LSP es el gravitino se realizaran los
  pasos \ref{item:7} y \ref{item:8} de la metodología.
\end{gravitinodm}

\begin{brpvlhc}
  En el marco de este proyecto se simularán dos correlaciones
  adicionales que existen en el modelo con ruptura bilineal de paridad R:
  Cuando el neutralino es la LSP, los decaimientos a tres cuerpos
  mediados por sfermiones con muones y electrones en los estados
  finales, están correlaciones con el ángulo de mezcla solar, y la
  longitud de decaimiento del neutralino está correlacionada con la
  diferencia de masa atmosférica.
\end{brpvlhc}

\begin{bbrpvlhc}
   Se estudiará además como estas predicciones, correlaciones y
   restricciones se mantienen o relajan cuando se consideran
   extensiones más del modelo BRpV en términos de operadores no
   renormalizables de dimensión cinco y violación de número leptónico.  
\end{bbrpvlhc}

\begin{darkmatter}
  Para el seesaw radiativo, en la literatura básicamente sólo se ha
  realizado el paso \ref{item:5}, aunque para algunas regiones del
  espacio de parámetros donde la partícula más liviana de paridad
  impar (LOP de sus siglas en inglés) es escalar se ha llegado hasta
  el paso \ref{item:8}~\cite{Bergman:2007pm}. En este proyecto haremos un
  estudio sistemático del modelo incluyendo todos los pasos de la
  metodología. Cómo el modelo está basado en una simetría $Z_2$,
  implementaremos el modelo en MicrOMEGAs~\cite{Belanger:2010gh} donde
  además de calcular la densidad de reliquia, y la sección eficaz
  WIMP-nucleón, se pueden obtener todas las secciones eficaces y
  amplitudes de decaimiento con el formato SLHA, lo que facilitará las
  simulaciones para el LHC.
\end{darkmatter}




A través de todo el proyecto se requiere de una infraestructura de
computación de alto rendimiento adecuada, que garantice una
disponibilidad permanente de poder de computo para las diferentes
simulaciones y programas computacionales que requiere el proyecto. El
grupo ha venido consolidando sus herramientas de computación con la
adquisición de un servidor Blade con capacidad para 8 nodos, aunque de
momento sólo tiene un nodo instalado. Toda esta experiencia le ha
permitido consolidar un grupo de desarrolladores y administradores de
software científico que ahora conforman La división de ciencias de la
computación del grupo la cual es ahora la encargada de administrar el
Centro Regional de Simulación y Cálculo Avanzado (CRESCA) y ofrece
servicio de computación científica.  Una de las metas prioritarias es
la unificar el poder de computo de los Grupos de Investigación de la
Universidad, como el servidor Blade del grupo, en un Grid
Institucional que la vez este conectado con Grid-Colombia. En el
proyecto solicitamos, además de los recursos para seguir mejorando la
infraestructura tecnológica a través de nodos Blade que soporten la
arquitectura CUDA para computación distribuida a través de GPU, la
contratación de un Administrador de Sistemas, que dedicaría parte de
su tiempo para la administración de CRESCA y la implementación del
Grid Institucional de computación de alto rendimiento.



\subsection{Actividades:                                   }
\label{sec:actividades}
Las actividades a desarrollar durante el proyectos consistirán en
pasantías, visitas y la organización de un taller sobre herramientas de
computación científica en física de altas energías.

\begin{enumerate}
\item Visita del Dr. Nicolás Bernal, por dos semanas para trabajar en
  la formulación del modelo más allá del BRpV con la implementación de
  operadores no-renormalizables de dimensión cinco con violación de
  número leptónico, y estudiar las implicaciones descritas en los
  objetivos~\ref{sec:objet-espec}.\ref{item:bbrpvlhc1}--\ref{item:bbrpvlhc2}. Así
  como la implementación de los cambios desarrollados al programa
  computacional SuSpect descritos en el
  objetivo~\ref{sec:objet-espec}.\ref{item:bbrpvlhc1}.~\ref{item:gravitinodm2}. 
  \label{item:act1}

\item En el mismo sentido se plantea una pasantía por dos semanas del
  Profesor Óscar Zapata a la Universidad de Bonn para trabajar estos
  temas y contar con la asesoría del profesor Herbert Dreiner quien es
  un experto en la implementación numérica de masas y mezclas de
  neutrinos en modelos supersimétricos con rotura trilineal de paridad
  R con términos bilineales inducidos a través de las ecuaciones del
  Grupo de la Renormalización~\cite{Dreiner:2011ft}.
\item Visita del Dr. Carlos Yaguna para asesorar al estudiante de
  maestría en el cálculo de la sección eficaz WIMP--nucleón a un bucle
  en el modelo seesaw radiativo, en especial en el caso en que la
  materia oscura corresponde a un escalar inerte. Con dicha visita
  también se avanzará en la implementación del programa para calcular
  los decaimientos del gravitino en un modelo supersimétrico con
  rotura trilineal de paridad R y términos bilineales inducidos
  detallados en el
  objetivo~\ref{sec:objet-espec}.~\ref{item:gravitinodm1}.

\item En el mismo sentido se plantea una pasantía de dos semanas del
  profesor Diego Restrepo a la Universidad de Münster para trabajar
  en estos temas.
\item Visita del Dr. Diego Aristizábal, por dos semanas para  calcular el
  background de rayos cósmicos de positrones y antiprotones y estudiar su
  influencia en las restricciones del los modelos planteados en la
  actividad~\ref{item:act1}. Colaborar en el desarrollo del
  objetivo~\ref{sec:objet-espec}.~\ref{item:darkmatter2} concerniente
  a la implementación de simetrías no Abelianas para simplificar el
  sector de Yukawa de los neutrinos derechos en el seesaw radiativo.

\item En el mismo sentido se plantea una pasantía del Profesor Luis
  Alfredo Muñoz a la Universidad de Liège para trabajar en estos temas y
  aprovechar la cercanía con las otras dos universidades de Alemania
  para avanzar en los demás objetivos del proyecto.
\item Pasantía por dos meses del estudiante de Doctorado en la
  Universidad de Bonn y la Universidad de Münster para trabajar en
  los tópicos detallados en la actividad~\ref{item:act1}.
\item Visita por dos semanas del Dr. Carlos Yaguna para participar en
  el evento \emph{Workshop on Computational High Energy Physics}, para
  desarrollarse en el ITM en octubre de 2012. Durante dicho evento el
  Dr. Yaguna realizará un mini--curso sobre el uso de MicrOMEGAs para
  el cálculo de la densidad de reliquia de materia en modelos de física
  más del Modelo Estándar.
\item Asistencia a evento en el área por algún miembro de la
  colaboración para presentar avances de resultados.
\end{enumerate}




\subsection{Resultados esperados:                          }
\begin{evaluacion}
  Definición clara de los resultados esperados.
\end{evaluacion}
\begin{instrucciones}
CODI:  Impacto y relevancia:
¿El proyecto permite la generación de conocimiento científico o aporta a la resolución de problemas concretos de la realidad? ¿Son suficientes y adecuados los mecanismos de comunicación y socialización de resultados? 

 COLCI: Formule los resultados directos verificables que se
alcanzarán con el desarrollo de los objetivos específicos del proyecto. Estos deben ser coherentes
con los objetivos y con la metodología planteada.



  \begin{enumerate}
  \item \textbf{Relacionados con la generación de conocimiento y/o nuevos desarrollos
 tecnológicos:} Incluye resultados/productos que corresponden a nuevo
 conocimiento científico o tecnológico o a nuevos desarrollos o adaptaciones de
 tecnología que puedan verificarse a través de publicaciones científicas,
 productos o procesos tecnológicos, patentes, normas, mapas, bases de datos,
 colecciones de referencia, secuencias de macromoléculas en bases de datos de
 referencia, registros de nuevas variedades vegetales, etc.
\item \textbf{Conducentes al fortalecimiento de la capacidad científica
  nacional:} Incluye resultados/productos tales como formación de
  recurso humano a nivel profesional o de posgrado (trabajos de grado
  o tesis de maestría o doctorado sustentadas y aprobadas),
  realización de cursos relacionados con las temáticas de los
  proyectos (deberá anexarse documentación soporte que certifique su
  realización), formación y consolidación de redes de investigación
  (anexar documentación de soporte y verificación) y la construcción
  de cooperación científica internacional (anexar documentación de
  soporte y verificación).
\item \textbf{Dirigidos a la apropiación social del conocimiento:}
  Incluye aquellos resultados/productos que son estrategias o medios
  para divulgar o transferir el conocimiento o tecnologías generadas
  en el proyecto a los beneficiarios potenciales y a la sociedad en
  general. Incluye tanto las acciones conjuntas entre investigadores y
  beneficiarios como artículos o libros divulgativos, cartillas,
  videos, programas de radio, presentación de ponencias en eventos,
  entre otros.
  \end{enumerate}

  Para cada uno de los resultados/productos esperados identifique (en
  los cuadros a continuación) indicadores de verificación (ej:
  publicaciones, patentes, registros, videos, certificaciones, etc.)
  así como las instituciones, gremios y comunidades beneficiarias,
  nacionales o internacionales, que podrán utilizar los resultados de
  la investigación para el desarrollo de sus objetivos, políticas,
  planes o programas:

\end{instrucciones}

\begin{instrucciones}
  Los impactos no necesariamente se logran al finalizar el proyecto, ni
con la sola consecución de los resultados/productos. Los impactos
esperados son una descripción de la posible incidencia del uso de los
resultados del proyecto en función de la solución de los asuntos o
problemas estratégicos, nacionales o globales, abordados. Generalmente
se logran en el mediano y largo plazo, como resultado de la aplicación
de los conocimientos o tecnologías generadas a través del desarrollo
de una o varias líneas de investigación en las cuales se inscribe el
proyecto. Los impactos pueden agruparse, entre otras, en las
siguientes categorías: sociales, económicos, ambientales, de
productividad y competitividad. Para cada uno de los impactos
esperados se deben identificar indicadores cualitativos o
cuantitativos verificables as\'\i:
\end{instrucciones}


Hemos entrado en una nueva era de la física que, de obtener los
resultados esperados, combinaría los descubrimientos de nuevas
partículas, a los cada vez mejor establecidos resultados de física de
neutrinos y observaciones cosmológicas sobre materia oscura. El
resultado de este proyecto es aportar a esta área de la ciencia con
nuevas propuestas de señales para ser buscadas en estos detectores
(subterráneos y en el espacio) y con la interpretación de los
resultados que surjan de ellos en términos de los modelos propuestos,
en los cuales se añaden partículas nuevas al Modelo Estándar de la
partículas elementales. Estos resultados se reflejarán en la
publicación de al menos tres artículos científicos en el área con la
participación de un estudiante de Doctorado, y en la presentación de
los resultados en al menos una conferencia internacional.


De especial importancia es el nuevo paradigma científico que surgirá
de la combinación de todos esos resultados experimentales. Por ejemplo
el descubrimiento en el LHC de la partícula escalar elemental predicha
por el Modelo Estándar, el Higgs, establecería finalmente las teorías
gauge con rompimiento espontáneo de simetría como el principio
fundamental para describir las interacciones subatómicas, redondeando
décadas de desarrollo científico.  El estudio detallado de las
propiedades del Higgs, acompañado posiblemente de señales
de nueva física, podría marcar el camino para encontrar el mecanismo
de generación de masas y mezclas de neutrinos, así como la
determinación de la partícula que compone la materia oscura del
Universo. A más tardar al finalizar esta década, se espera tener
respuesta a todos estos interrogantes. Todo esto tendrá un impacto en la
enseñanza de la física a todos los niveles. Además el descubrimiento
del Higgs, que sería la primera partícula escalar elemental, daría un
mejor fundamento teórico a los modelos inflacionarios en cosmología y
a la interpretación de la energía oscura como la causante de la
expansión acelerada del Universo. El otro escenario posible en el que
no se encuentre al Higgs del Modelo Estándar, debe dar lugar a datos
experimentales suficientes para dilucidar cual es realmente el
mecanismo de rotura de la simetría electrodébil. Nuestro grupo es el
más directamente llamado a difundir estos avances en nuestro entorno social,
como lo ha venido haciendo a través de conferencias y cursos de
extensión en los últimos años. Es importante que nuestro país siga
participando en el desarrollo de la física fundamental, no sólo con la
participación de grupos teóricos como el nuestro, sino también con
grupos experimentales de física de altas energías como lo viene
haciendo en las colaboraciones ATLAS y CMS del LHC con grupos de la
Universidad Antonio Nariño y de los Andes respectivamente. Con ellos,
y con los otros grupos teóricos del país, hemos venido colaborando y
organizando congresos en el área en los últimos años para consolidar
esta área de investigación en el país.

El principal aporte de un Grupo como el nuestro al desarrollo del país
es la formación de talento humano con capacidad de hacer investigación
científica al más alto nivel. Para ello es prioritario que nuestro
Grupo siga produciendo productos de gran impacto en la comunidad
mundial de física de altas energías con participación de nuestros
estudiantes de pregrado y posgrado.  Aunque de momento los doctores
que formamos son rápidamente reabsorbidos en el ámbito académico,
esperamos que a futuro, como pasa en otros países donde profesionales
de este tipo son muy apreciados en empresas de innovación tecnológica,
los nuestros puedan llegar a hacer aportes significativos a otros
sectores de la sociedad. También hemos logrado capacitar nuestro
equipo técnico en herramientas de computación científica y de
administración de redes en Linux.


De ser aprobado, este proyecto nos permitiría participar en esta
excitante era del desarrollo científico que coincide con
los primeros años de funcionamiento del LHC.  Se espera que entre los
resultados de los próximos años LHC, no solo esté el del
descubrimiento del Higgs, sino también de alguna señal de física más
allá del Modelo Estándar que explique los problemas fenomenológicos y
teóricos del Modelo Estándar. En los próximos años también se espera
que los experimentos de detección directa, o indirecta a través de
rayos cósmicos, entreguen una evidencia definitiva de materia oscura.

Se espera que una vez realizados con éxito los objetivos trazados en
este proyecto, se amplíen las líneas de investigación 


\begin{gravitinodm}
  Se espera que una vez realizados con éxito los objetivos trazados en
  este proyecto, se establezca cuáles son las implicaciones que
  generan el abandono de la hipótesis de leptogénesis y la no
  universalidad de las masas de los gauginos, sobre las masas de los
  neutrinos en modelos con violación de paridad R y materia oscura de
  gravitinos. Es decir, determinar si es posible obtener un modelo con
  operadores de rotura de paridad R hasta dimensión cinco, que puedan
  explicar la densidad de materia oscura con gravitinos, dar cuenta de
  los datos de física de neutrinos y sea consistente con las
  restricciones provenientes de rayos cósmicos.
\end{gravitinodm}

\begin{bbrpvlhc}
....
\end{bbrpvlhc}

\begin{brpvlhc}
  ..
\end{brpvlhc}

\begin{darkmatter}
  ....
\end{darkmatter}

%%%Para adicionar a los cambios que haga Oscar
Uno de los resultados más importante que se espera obtener de este
proyecto es el establecimiento de modelos supersimétricos difíciles de
encontrar en el LHC por la degradación de la señal de energía y por
alta actividad hadrónica que den una explicación satisfactoria par las
masas y mezclas de los  neutrinos y tengan al gravitino como candidato
de materia oscura con implicaciones concretas en experimentos de rayos
cósmicos.  

Para maximizar el uso intensivo de computación de alto rendimiento
se vinculará a la Universidad de Antioquia a la red de computación
distribuida \emph{Grid Colombia}, asignando prioridades a los usuarios
de la Universidad de Antioquia de los diferentes cluster de
computadores con los que cuenta la Institución detallados en el
presupuesto. 

\subsection{Cronograma:                                    }
\begin{evaluacion}
  Fechas para la entrega de informes trimestrales
\end{evaluacion}

\begin{instrucciones}
Duración máxima 18 meses
  ¿La secuencia de actividades se adecúa a las fases de desarrollo del proyecto? ¿La duración en cada una de las etapas es apropiada y garantiza el cumplimiento del objetivo?
\end{instrucciones}


\begin{itemize}
\item \textbf{meses 1-3:} Revisión bibliográfica sobre cálculos de
  secciones eficaces promediadas térmicamente, programas
  computacionales para cálculos de secciones eficaces y amplitudes de
  decaimientos, propiedades de los detectores y filtros de selección
  de datos en el LHC, propagación de los diferentes tipos de rayos
  cósmicos a través de la galaxia.  En esta fase también se formarán los
  estudiantes  de maestría Doctorado en los tópicos específicos relativos a la
  investigación, así como motivarle a tomar parte activa en todas las
  fases de desarrollo del proyecto.

\item \textbf{meses 3-6} Durante está fase desarrollaremos las
  herramientas computacionales necesarias para el proyecto, y se
  desarrollarán los cálculos analíticos necesarios.

\begin{brpvlhc}
  Con colaboradores internacionales de Brasil y Valencia hemos desarrollado un
  programa computacional en PYTHIA para calcular el nivel de precisión
  con el que se puede llegar a medir en el LHC la correlación entre el
  ángulo de mezcla atmosférico de neutrinos y el cociente decaimientos
  a dos cuerpos: $\tilde\chi_1^0\to W^\pm\mu^\mp$ sobre
  $\tilde\chi_1^0\to W^\pm\tau^\mp$, cuando el $W^\pm$ decae
  hadrónicamente.  En esta fase se modificará el programa para
  implementar las nuevas correlaciones propuestas.
\end{brpvlhc}

\begin{gravitinodm}
  Aprovechando la experiencia adquirida con Carlos Yaguna
  implementando los decaimientos a tres~\cite{Choi:2010xn,Choi:2010jt}
  y dos cuerpos\cite{Restrepo:2011rj} del gravitino en el
  BRpV~\cite{Choi:2010xn,Choi:2010jt} se realizará un programa para
  implementar todos los decaimientos a tres cuerpos, dos cuerpos y a
  un bucle para el modelo con ruptura de paridad R a través de
  términos trilineales con términos bilineales inducidos.

  Para el caso en que el gravitino es la LSP debemos modificar el
  programa para calcular las correlaciones con base en la NLSP
  teniendo en cuenta los nuevos canales en las cuales la NLSP puede
  decaer al gravitino.
\end{gravitinodm}



\begin{darkmatter}
  Para el seesaw radiativo, implementaremos el modelo en MicrOMEGAs.
  escribiremos el Lagrangiano del modelo el lenguaje de programación
  LanHEP~\cite{Semenov:2008jy}. Con este programa podremos generar las
  cuatro tablas en el formato de CalcHEP~\cite{Pukhov:2004ca}, el cual
  podemos usar para calcular las secciones eficaces y las amplitudes
  de decaimiento del modelo.  Como MicrOMEGAs usa internamente
  CalcHEP, también podremos calcular la densidad de reliquia y la
  sección eficaz WIMP-nucleón del modelo, implementando las
  correcciones a un bucle necesarias para una mejor implementación de
  las restricciones provenientes de los resultados experimentales de
  detección directa. Debemos crear también una interfase en Python
  para hacer los llamados a los diferentes cálculos y para implementar
  las regiones del espacio de parámetros que son compatibles con los
  datos de oscilaciones de neutrinos. Paralelamente debemos hacer
  cálculos analíticos de las secciones eficaces promediadas
  térmicamente para estar seguros que los resultados de MicrOMEGAs son
  correctos.
\end{darkmatter}

\item \textbf{meses 6-12} En esta parte se obtendrán y analizarán los
  resultados numéricos de los modelos bajo consideración. El
  estudiante de doctorado realizará su pasantía doctoral en un Grupo
  de Investigación europeo. 

\item \textbf{meses 12-18} En esta parte se prepararán los artículos
  para publicación con base en los resultados obtenidos. Uno sobre el
  nivel de precisión con la que se pueden medir correlaciones nuevas
  del modelo de ruptura bilineal de paridad R en el LHC, otro sobre
  la formulación de un modelo consistente de ruptura trilineal de
  paridad $R$ con operadores de violación de número leptónico, con el
  gravitino como materia oscura inestable, y otro sobre las
  implicaciones fenomenológicas del seesaw radiativo en aceleradores
  de partículas y en experimentos de detección directa de materia
  oscura. 

  Se presentarán los resultados en al menos una conferencia
  internacional y se preparará el informe final del proyecto.
\end{itemize}

\subsection{Equipo de trabajo:}
\begin{tabular}{|p{3.7cm}|p{5cm} |l|l|}\hline
Recurso humano (rol)& Responsabilidad& Unidades (días, meses)& \# de unidades\\\hline
Diego Restrepo&Investigador Principal&(0,18)&1\\\hline
Óscar Zapata y Luis A. Muñoz & Coinvestigadores &(0,18)&2\\\hline
Jose David Ruiz & Estudiante de Maestría &(0,18)&1 \\\hline
Mauricio Velásquez & Estudiante de Doctorado &(0,18)&1\\\hline
Diego Aristizábal,
Nicolás Bernal, 
Carlos Yaguna & Investigadores diáspora &(0,18)&3\\\hline
Auxiliar de Sistemas & Optimizar programas para correr en paralelo, 
                       adaptar código para usar arquitectura CUDA en GPUs, 
                       conectar nodos existentes a través de tecnología 
                       grid y unirse oficialmente a Grid Colombia &(0,18)&1\\\hline
\end{tabular}

\subsection{Presupuesto}
\begin{instrucciones}
  Por favor siga el modelo de la tabla que encontrará en este punto
  para la presentación del presupuesto.

  Hasta 150 millones.
  
  * Apoyo logístico
  * Materiales ..., insumos ... requeridos para la realización del proyecto.
  * Servicios tecnológicos: (..., Simulaciones, ...)
  * Gastos de transporte y viáticos
  * Costos operativos y de administración.
  * Difusión de resultados

\end{instrucciones}
\begin{tabular}{|p{9cm}|l|l|l|}\hline
\gdef\EFP#1{\FPeval\RR{clip(#1)}\numprint{\RR}}%%\EFP:eval #1  
\gdef\RFP#1#2{\FPeval\RR{clip(#2)}\FPround\RR\RR{#1}\RR}%%\RFP:#2rounded to#1
%%\RR is the output of either \EFP or \RFP
  \multirow{2}{*}{Rubros}&\multicolumn{2}{c}{Fuentes}\vline&\multirow{2}{*}{Total}\\
  \cline{2-3} & Colciencias & Contrapartida & \\\hline 
Investigador Principal (10h/semana)      &                              &$\EFP{29000000}\xdef\Cc{\RR}$&$\EFP{Cc}\xdef\Cd{\RR}$\\\hline
Coinvestigador (10h/semana)              &                              &$\EFP{21000000}\xdef\Dc{\RR}$       &$\EFP{Dc}\xdef\Dd{\RR}$\\\hline
Coinvestigador (10h/semana)              &                              &$\EFP{20000000}\xdef\Ec{\RR}$       &$\EFP{Ec}\xdef\Ed{\RR}$\\\hline
Estudiante de doctorado (Est. Instructor)&$\EFP{27000000}\xdef\Fb{\RR}$ &$\EFP{14400000}\xdef\Fc{\RR}$&$\EFP{(Fb+Fc)}\xdef\Fd{\RR}$\\\hline
Estudiante de Maestría (Est. Instructor) &$\EFP{18000000}\xdef\Gb{\RR}$ &$\EFP{14400000}\xdef\Gc{\RR}$&$\EFP{(Gb+Gc)}\xdef\Gd{\RR}$\\\hline
Auxiliar de Sistemas                      &$\EFP{18000000}\xdef\Hb{\RR}$ &$\EFP{       0}\xdef\Hc{\RR}$&$\EFP{(Hb+Hc)}\xdef\Hd{\RR}$\\\hline
Nodo blade server                        &$\EFP{10000000}\xdef\Ib{\RR}$ &$\EFP{       0}\xdef\Ic{\RR}$&$\EFP{(Ib+Ic)}\xdef\Id{\RR}$\\\hline
Papelería e impresiones                  &$\EFP{ 5000000}\xdef\Jb{\RR}$ &$\EFP{       0}\xdef\Jc{\RR}$&$\EFP{(Jb+Jc)}\xdef\Jd{\RR}$\\\hline
Cluster de Simulación y Cálculo avanzado %
(Rack 22 nodos); servidor Blade con 1    %
nodo; cluster de matemáticas             %
(Rack 7 nodos)                           &$\EFP{0}\xdef\Kb{\RR}$        &$\EFP{100000000}\xdef\Kc{\RR}$&$\EFP{(Kb+Kc)}\xdef\Kd{\RR}$\\\hline
Asistencia congresos                     &$\EFP{0}\xdef\Lb{\RR}$        &$\EFP{ 5000000}\xdef\Lc{\RR}$&$\EFP{(Lb+Lc)}\xdef\Ld{\RR}$\\\hline
Pasantías y asistencia congresos         &$\EFP{33500000}\xdef\Mb{\RR}$ &$\EFP{ 7000000}\xdef\Mc{\RR}$&$\EFP{(Mb+Mc)}\xdef\Md{\RR}$\\\hline
Visitantes diáspora                      &$\EFP{34000000}\xdef\Wb{\RR}$ &$\EFP{ 7000000}\xdef\Wc{\RR}$&$\EFP{(Wb+Wc)}\xdef\Wd{\RR}$\\\hline
Administración 3\%                       &$\EFP{ 4500000}\xdef\Xb{\RR}$ &$\EFP{       0}\xdef\Xc{\RR}$&$\EFP{(Xb+Xc)}\xdef\Xd{\RR}$\\\hline
TOTAL (en pesos) &$\EFP{(Fb+Gb+Hb+Ib+Jb+Kb+Lb+Mb+Wb+Xb)}\xdef\Yb{\RR}$ &$\EFP{(Cc+Dc+Ec+Fc+Gc+Hc+Ic+Jc+Kc+Lc+Mc+Wc+Xc)}\xdef\Yc{\RR}$%
                 &$\EFP{(Cd+Dd+Ed+Fd+Gd+Hd+Id+Jd+Kd+Ld+Md+Wd+Xd)}\xdef\Yd{\RR}$\\\hline
Total en porcentaje &$\RFP{2}{(Yb/Yd)*100}\xdef\Zb{\RR}$ &$\RFP{2}{(Yc/Yd)*100}\xdef\Zb{\RR}$ &$\RFP{2}{(Yd/Yd)*100}\xdef\Zb{\RR}$\\\hline
\end{tabular}

\subsection{Plan de acción.}
Ver Tabla 1 al final
\begin{evaluacion}
  Fechas para la entrega de informes trimestrales
\end{evaluacion}
\begin{sidewaystable}[H]
\footnotesize
\caption{3.14 Plan de acción}  % title name of the table
\centering
\begin{tabular}{|p{2cm}|l|l|l|l|l|l|l|l|l|l|l|}\hline
  \multirow{2}{*}{Objetivo} & \multirow{2}{*}{Estrategia} & \multirow{2}{*}{Indicador}  & \multicolumn{9}{|c|}{MQtas}\\
\cline{4-12} 
& & & Bimestre 1 &Bimestre 2 & Bimestre 3& Bimestre 4& Bimestre 5& Bimestre 6& Bimestre 7& Bimestre 8& Bimestre 9\\\hline 
&&&&&&&&&&&\\\hline
\end{tabular}
\label{tab:LPer}
\end{sidewaystable}

%%% Local Variables: 
%%% mode: latex
%%% TeX-master: "proyecto"
%%% End: 
